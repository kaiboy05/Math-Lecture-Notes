\documentclass[a4paper]{article}

\def\npart {Year 1}
\def\nterm {First Term}
\def\nyear {2018}
\def\nlecturer {Kevin Buzzard}
\def\ncourse {Foundation of Analysis}

\input{header}

\begin{document}

\maketitle
{\small
  \noindent\textbf{M1F}\\
}

\tableofcontents

\setcounter{section}{-1}

\section{Introduction}
In this course...

\section{Propositions, Sets and Numbers}
The propositions are like easy logic, and then a few sets and number concept will be discussed.

\subsection{Propositions}
\begin{defi}[Proposition]
	A \emph{proposition} is a \textbf{True} or \textbf{False} statement.
\end{defi}

\begin{eg}\leavevmode
	\begin{itemize}
		\item $2 + 2 = 4$
		\item $2 + 2 = 100000000$
		\item Fermat's Last Theorm
		\item Riemann Hypothesis
	\end{itemize}
\end{eg}

There are some propositions that we don't know they are true or false, like Riemann hypothesis. However, in \emph{classical mathematics}, mathematics of M1F, \textbf{every} proposition is either true or not. We are just not sure about some of them.

There are also some examples of things which are \textbf{not} propositions:

\begin{eg}\leavevmode
	\begin{itemize}
		\item $2 + 2$
		\item $2 = 2 = 4$
	\end{itemize}
	The first example is a number, but not proposition. It is not 'true' or 'false', it is 4.
	The second example doesn't even make sense. It is not a mathematical object.
\end{eg}

\subsection{Notation of proposition}
There are few connectives between propositions, they are \textbf{and}, \textbf{or}, \textbf{not}, \textbf{implies}, \textbf{if and only if}

\begin{defi}[And]
If $P$ and $Q$ are propositions, "$P$ \emph{and} $Q$" is a proposition and can be written as $P \land Q$. $P \land Q$ are true when \emph{both} $P$ and $Q$ are true.
\end{defi}

We can see the relation of $P \land Q$, $P$, and $Q$ by the truth table.

\begin{center}
	\begin{tabular}{|c|c|c|}
		\hline
		$P$ & $Q$ & $P \land Q$\\
		\hline
		$T$ & $T$ & $T$\\
		\hline
		$T$ & $F$ & $F$\\
		\hline
		$F$ & $T$ & $F$\\
		\hline
		$F$ & $F$ & $F$\\
		\hline
	\end{tabular}
\end{center}

\begin{eg}
$(2 + 2 = 4) \land (2 + 2 = 5)$ is false, since $2 + 2 = 5$ is false.
\end{eg}

\begin{defi}[Or]
If $P$ and $Q$ are propositions, "$P$ \emph{or} $Q$" is a proposition and can be written as $P \lor Q$. $P \lor Q$ are true when \emph{either} $P$, $Q$ or \emph{both} are true.
\end{defi}

We can see the relation of $P \lor Q$, $P$, and $Q$ by the truth table.

\begin{center}
	\begin{tabular}{|c|c|c|}
		\hline
		$P$ & $Q$ & $P \lor Q$\\
		\hline
		$T$ & $T$ & $T$\\
		\hline
		$T$ & $F$ & $T$\\
		\hline
		$F$ & $T$ & $T$\\
		\hline
		$F$ & $F$ & $F$\\
		\hline
	\end{tabular}
\end{center}

\begin{eg}
$(2 + 2 = 4) \lor (2 + 2 = 5)$ is false, since $2 + 2 = 4$ is true.
\end{eg}

\begin{defi}[Not]
If $P$ is proposition, "not $P$" is a proposition and can be written as $\neg P$. $\neg P$ is the proposition which is "the opposite of $P$". If $P$ is true then $\neg P$ is false, and if $P$ is false then $\neg P$ is true.
\end{defi}

We can see the relation of $\neg P$ and $P$ by the truth table.

\begin{center}
	\begin{tabular}{|c|c|}
		\hline
		$P$ & $\neg P$\\
		\hline
		$T$ & $F$\\
		\hline
		$F$ & $T$\\
		\hline
	\end{tabular}
\end{center}

\begin{eg}
Let $P$ be the Riemann hypothesis, then $P \lor \neg P$ is true, because in classical mathematics, the Riemann hypothesis is either true or false.
\end{eg}

\begin{defi}[Implies]
If $P$ and $Q$ are propositions, "$P$ \emph{implies} $Q$" is a proposition and can be written as $P \implies Q$. $P \implies Q$ means if $P$ is true, then $Q$ is true as well.
\end{defi}

We can see the relation of $P \implies Q$, $P$, and $Q$ by the truth table.

\begin{center}
	\begin{tabular}{|c|c|c|}
		\hline
		$P$ & $Q$ & $P \implies Q$\\
		\hline
		$T$ & $T$ & $T$\\
		\hline
		$T$ & $F$ & $F$\\
		\hline
		$F$ & $T$ & $T$\\
		\hline
		$F$ & $F$ & $T$\\
		\hline
	\end{tabular}
\end{center}

The only time that $P \implies Q$ is false is when $P$ is true and $Q$ is false.

\begin{eg}
$(2 + 2 = 4) \implies (2 + 2 = 5)$ is false, but $(2 + 2 = 5) \implies (2 + 2 = 4)$ is true.
\end{eg}

\begin{notation}
$Q \Longleftarrow P$ is defined to be $P \implies Q$.
\end{notation}

\begin{defi}[if and only if]
If $P$ and $Q$ are propositions, "$P$ \emph{if and only if} $Q$" is a proposition and can be written as $P \iff Q$. $P \iff Q$ is true when $P$ and $Q$ have the \emph{same} truth value.
\end{defi}

We can see the relation of $P \implies Q$, $P$, and $Q$ by the truth table.

\begin{center}
	\begin{tabular}{|c|c|c|}
		\hline
		$P$ & $Q$ & $P \iff Q$\\
		\hline
		$T$ & $T$ & $T$\\
		\hline
		$T$ & $F$ & $F$\\
		\hline
		$F$ & $T$ & $F$\\
		\hline
		$F$ & $F$ & $T$\\
		\hline
	\end{tabular}
\end{center}

$\iff$ is the proposition version of $=$ for numbers. If $x$ and $y$ are equal numbers, we write $x = y$, but if $P$ and $Q$ are propositions with the same truth value, we write $P \iff Q$.

\begin{eg}\leavevmode
	\begin{itemize}
		\item $(P \implies Q) \iff (Q \Longleftarrow P)$ is always true.
		\item $P \iff (\neg P)$ is always false.
	\end{itemize}
\end{eg}

\subsection{Theorm of propositions}
\begin{thm}[Relation of \emph{not, in} and \emph{or}]
Let $P$ and $Q$ be propositions,
$$(\lnot P) \lor (\lnot Q) \iff \lnot(P \land Q)$$
\end{thm}
\begin{proof}
Consider the truth table,
\begin{center}
	\begin{tabular}{|c|c|c|c|}
		\hline
		$P$ & $Q$ & $(\lnot P) \lor (\lnot Q)$ & $\lnot(P \land Q)$\\
		\hline
		$T$ & $T$ & $F \lor F \iff F$ & $\lnot(T) \iff F$\\
		\hline
		$T$ & $F$ & $F \lor T \iff T$ & $\lnot(F) \iff T$\\
		\hline
		$F$ & $T$ & $T \lor F \iff T$ & $\lnot(F) \iff T$\\
		\hline
		$F$ & $F$ & $T \lor T \iff T$ & $\lnot(F) \iff T$\\
		\hline
	\end{tabular}
\end{center}
We can see that the truth values of proposition $(\lnot P) \lor (\lnot Q)$ and $\lnot(P \land Q)$ are always the same.
$$\therefore (\lnot P) \lor (\lnot Q) \iff \lnot(P \land Q)$$
\end{proof}

\subsection{Sets}
\begin{defi}[Set]
	A \emph{set} is a collection of stuff. The things in a set $X$ are called the \emph{elements} of $X$.
\end{defi}
Note that there is a more rigorous definition of a set. The more rigorous one depends on which axiomatic foundation using for mathematics. If set theory is the foundation, the definition of a set will be "\textbf{Everything is a set.}".

\subsection{Basic notation for sets.}

\begin{notation}
We use $\{$ and $\}$ to denote sets.
\end{notation}

\begin{eg}\leavevmode
	\begin{itemize}
		\item $\{1, 2, 3\}$ is a set.
		\item $\{$ me, you, the desk in my office $\}$ is a set.
		\item $\{\}$ is a set. It exists, but it has no elements.
		\item $\{1, 2, 3, 2\}$ is a set.
		\item $\{1, 2, 3, 4, 5, ...\}$ is a set, and it is an infinite set.
	\end{itemize}
\end{eg}

We use the symbol $\in$ to denote set membership. If $a$ is a thing (e.g. a number) and $X$ is a set, then $a \in X$ is a proposition. The proposition $a \in X$ is true exactly when $a$ is in set $X$.

\begin{eg}\leavevmode
	\begin{itemize}
		\item $2 \in \{ 1, 2, 3 \}$. This means $2$ is an element of set $\{ 1, 2, 3 \}$.
		\item $x \in \{ \}$ makes mathematical sense, but it is a false statement.
	\end{itemize}
\end{eg}

\begin{notation}
$\{ \}$ has no elements, which is called the \emph{empty set}. We use $\varnothing$ to notate an empty set.
\end{notation}

\subsection{Fundamental fact about equality of sets}

\begin{defi}[Equality of sets]
\[
	X = Y \iff (\forall a \in \Omega, a \in X \iff a \in Y)
\]
It means two sets are equal if and only if they have the same elements.
\end{defi}

\begin{eg}
$\{ 1, 2, 3 \}$ and $\{ 1, 2, 3, 2 \}$ are equal.
\end{eg}

Fundamental fact above is the rule for sets. If we need to count things, we can use other things, like multisets, lists, or sequences, instead of sets.

\subsection{Notation of sets}

\subsubsection{Subsets}
\begin{notation}
We use $\subseteq$ to denote subsets. $X \subseteq Y$ is a proposition saying that $X$ is a subset of $Y$.
\end{notation}

\begin{defi}[Subset]
\[
	X \subseteq Y \iff (\forall a \in \Omega, a \in X \implies a \in Y)
\]
It means X is a subset of Y when every elements of X is also an element of Y.
\end{defi}

\begin{eg}\leavevmode
	\begin{itemize}
		\item $\{ 1, 2 \} \subseteq \{ 1, 2, 3 \}$, since elements of set $\{ 1, 2\}$, $1$ and $2$ are both inthe set $\{ 1, 2, 3 \}$.
		\item If $a$ is my left shoe, $b$ is my right hand, and $c$ is my mother, then $\{ a, b \} \subseteq \{ a, b, c \}$
	\end{itemize}
\end{eg}

\begin{notation}
$X \supseteq Y$ means $X \subseteq Y$.
\end{notation}

\begin{thm}[Equality and subsets]
If $X$ and $Y$ are sets, then
\[
	X = Y \iff (X \subseteq Y \land Y \subseteq X)
\]
\end{thm}

\begin{proof}
From $X \subseteq Y$, we can deduce 
\[
	a \in X \implies a \in Y \tag{1}
\]
And from $Y \subseteq X$, we can deduce
\[
	a \in Y \implies a \in X \tag{2}
\]
From (1) and (2), we can deduce that $$a \in Y \iff a \in X$$which is definition of $X = Y$
\[
	\therefore (X \subseteq Y \land Y \subseteq X) \implies X = Y \tag{a}
\]
Similarly, From $X = Y$, we can deduce
\[
	a \in Y \iff a \in X
\]
And it is equivalent to
\begin{align*}
	a \in Y &\implies a \in X\\
	a \in X &\implies a \in Y
\end{align*}
which are definition of $Y \subseteq X$ and $X \subseteq Y$.
\[
	\therefore X = Y \implies (X \subseteq Y \land Y \subseteq X) \tag{b}
\]
With (a) and (b), we can conclude that,
\[
	X = Y \iff (X \subseteq Y \land Y \subseteq X)
\]
\end{proof}

\subsection{Important sets}
\begin{eg}\leavevmode
	\begin{itemize}
		\item $\Z$ Integers
		\item $\Q$ Rational numbers
		\item $\R$ Real numbers
		\item $\C$ Complex numbers
	\end{itemize}
\end{eg}

\begin{defi}[Integers $\Z$]
$$\Z = \{ ..., -3, -2, -1, 0, 1, 2, 3, ...\}$$
\end{defi}
There is a problem of unconsistent of natural numbers $\N$. Someone defined it as $$\N = \{ 0, 1, 2, 3, ...\}$$Someone defined it as $$\N = \{ 1, 2, 3, ...\}$$
In M1F, we will not use $\N$. Instead, we will use the following notations.
\begin{notation}
$$\Z _{\geq 0} = \{ 0, 1, 2, 3, ...\}$$
$$\Z _{\geq 1} = \{ 1, 2, 3, ...\}$$
\end{notation}

For set $\R$, there are some special notations.
\begin{notation}
Let $a$ and $b$ be real numbers,
$$[a,b] = \{x \in \R \mid a \leq x \land x \leq b\}$$
$$(a,b) = \{x \in \R \mid a < x \land x < b\}$$
$$[a,\infty) = \{x \in \R \mid a \leq x\}$$
\end{notation}

\subsubsection{Universes}
\begin{notation}
We use \emph{universe} $\Omega$ to denote the set consisting of all the stuff we are interested in.
\end{notation}
\emph{Universe} means the set we are considering. For example, $\Omega$ could be a set of real numbers, or complex numbers. It depends on what we are considering.

\subsubsection{For all}
\begin{notation}
We use $\forall$ to say for all in mathematics.
\end{notation}

\begin{eg}
$\forall a \in \Z, 2a$ is even.\\
This means ``For all integers $a$, $2a$ is an even number''.
\end{eg}

\subsubsection{There exists}
\begin{notation}
We use $\exists$ to say there exists inn mathematics.
\end{notation}

\begin{eg}
$\exists a \in \Z, a$ is even.\\
This means ``There exists an integer $a$, which is an even number''.
\end{eg}

\subsubsection{Union}
\begin{defi}[Unions]
\[
	\forall a \in \Omega, a \in X \cup Y \iff a \in X \lor a \in Y
\]
It means the \emph{union} of $X$ and $Y$, $X \cup Y$ is all the stuff in either $X$, \emph{or} $Y$, \emph{or both}.
\end{defi}
\begin{eg}
Let $X = \{ 1, 2, 3 \}$ and $Y= \{ 3, 4, 5 \}$,\\
then $X \cup Y = \{ 1, 2, 3, 4, 5 \}$.
\end{eg}
We have some notations for intersection of large numbers of sets.\\
Let us define $I = \Z_{\geq1} = \{1,2,3,...\}$. For every $i \in I$, we have a set of real numbers $X_i \subseteq \R$.
\begin{notation}
$$\bigcup^\infty_{i=1} X_i = \{a \in \Omega \mid \exists i \in \Z_{\geq1}, a \in X_i\}$$
$$\bigcup_{i \in I} X_i = \{a \in \Omega \mid \exists i \in I, a \in X_i\}$$
\end{notation}

\begin{eg}
Let $I = \R$.
If $i \in I$, and let $X_i = \{i\}$.
What is $\bigcup_{i \in I} X_i$?
$$\bigcup_{i \in I} X_i = \{a \in \R \mid \exists i \in I, a \in X_i\}$$
$$\therefore \bigcup_{i \in I} X_i \subseteq \R$$
Let $a \in \R$,
\[
	a \in X_a = \{a\} \tag{by definition}
\]
$\therefore \exists i \in I = \R$ such that $a \in X_i = \{i\}$ when $i = a$.
\end{eg}

\subsubsection{Intersection}
\begin{defi}[Intersection]
\[
	\forall a \in \Omega, a \in X \cap Y \iff a \in X \land a \in Y
\]
It means the \emph{intersection} of $X$ and $Y$, $X \cap Y$ is all the stuff in \emph{both} $X$, \emph{and} $Y$.
\end{defi}
\begin{eg}
Let $X = \{ 1, 2, 3 \}$ and $Y= \{ 3, 4, 5 \}$,\\
then $X \cap Y = \{ 3 \}$.
\end{eg}
We have some notations for intersection of large numbers of sets.\\
Let us define $I = \Z_{\geq1} = \{1,2,3,...\}$. For every $i \in I$, we have a set of real numbers $X_i \subseteq \R$.
\begin{notation}
$$\bigcap^\infty_{i=1} X_i = \{a \in \Omega \mid \forall i \in \Z_{\geq1}, a \in X_i\}$$
$$\bigcap_{i \in I} X_i = \{a \in \Omega \mid \forall i \in I, a \in X_i\}$$
\end{notation}

\begin{eg}
What is $\bigcap_{i=1}^\infty X_i$, where $X_i = [-i, i]$?\\
$\because X_1 \subseteq X_2 \subseteq X_3 \subseteq ...$, real numbers in all the $X_i$ are the real numbers in $X_1$.\\
$\therefore \bigcap_{i=1}^\infty X_i = X_1$
\end{eg}


\subsubsection{Complements}
\begin{defi}[Complements]
\[
	\forall a \in \Omega, a \in X^c \iff \lnot(a \in X)
\]
It means if $X$ is a subset of $\Omega$, then its \emph{complement} $X^c$ is the set whose elements are all the things in $\Omega$ which are not in $X$.
\end{defi}
\begin{eg}
If our universe $\Omega$ is $\Z$, the integers, and if $X$ is the set of even integers, then its \emph{complement} $X^c$ is the set of odd numbers.
\end{eg}
\begin{notation}
$a \notin X$ is defined to be $\lnot(a \in X)$, since $a$ is not an element of $X$ is also a proposition.
\end{notation}

\subsection{Notation of sets with certain property}
Let $X$ be the set of \emph{integers}, and we want to consider the subset of $X$ consisting of positive integers. We can write the subset as: 
$$\{ a \in X \mid a > 0\}$$
The line in the middle is pronounced ``'such that". So the full statement can be read as ``the elements $a$ of $X$ such that $a > 0$".

\subsection{Theorm of sets}
\begin{thm}[A theorm of complement]
Let $X$ and $Y$ be sets.\\
If $X, Y \subseteq \Omega$, $$(X \cup Y = \Omega) \land (X \cap Y = \varnothing) \implies X = Y^c$$
\end{thm}
\begin{proof}
Let $a \in \Omega$,
$P$ be proposition $a \in X$ and $Q$ be proposition $a \in Y$,
\begin{align*}
	&&a \in X \cup Y &\iff (a \in X) \lor (a \in Y) \tag{Union definition}\\
	&\therefore &a \in X \cup Y &\iff P \lor Q\\
	&\because &X \cup Y = \Omega \\
	&\therefore &a \in X \cup Y &\iff \top\\
	&&P \lor Q &\iff \top\\
	&&a \in X \cap Y &\iff (a \in X) \land (a \in Y) \tag{Intersection definition}\\
	&\therefore &a \in X \cup Y &\iff P \land Q\\
	&\because &X \cup Y = \varnothing\\
	&\therefore &a \in X \cup Y &\iff \bot\\
	&&P \land Q &\iff \bot\\
	&&\lnot (P \land Q) &\iff \top\\
	&\therefore &P \lor Q &\iff \lnot (P \land Q)\\
	&\therefore &P &\iff \lnot Q\\
	&& a \in X &\iff \lnot(a \in Y)\\
	&& a \in X &\iff a \in Y^c \tag{Complement definition}\\
	&\therefore &X &= Y^c
\end{align*}
$$(X \cup Y = \Omega) \land (X \cap Y = \varnothing) \implies X = Y^c$$
\end{proof}

\subsection{More about sets}


























\end{document}