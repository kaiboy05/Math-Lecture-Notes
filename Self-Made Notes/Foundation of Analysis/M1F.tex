\documentclass[a4paper]{article}

\def\npart {Year 1}
\def\nterm {First Term}
\def\nyear {2018}
\def\nlecturer {Kevin Buzzard}
\def\ncourse {Foundation of Analysis}

\makeatletter
\ifx \nauthor\undefined
  \def\nauthor{Chester Wong}
\else
\fi

\author{Based on lectures by \nlecturer \\\small Notes taken by \nauthor}
\date{\nterm\ \nyear}

\usepackage{alltt}
\usepackage{amsfonts}
\usepackage{amsmath}
\usepackage{amssymb}
\usepackage{amsthm}
\usepackage{booktabs}
\usepackage{caption}
\usepackage{enumitem}
\usepackage{fancyhdr}
\usepackage{graphicx}
\usepackage{mathdots}
\usepackage{mathtools}
\usepackage{microtype}
\usepackage{multirow}
\usepackage{pdflscape}
\usepackage{pgfplots}
\usepackage{siunitx}
\usepackage{slashed}
\usepackage{tabularx}
\usepackage{tikz}
\usepackage{tkz-euclide}
\usepackage[normalem]{ulem}
\usepackage[all]{xy}
\usepackage{imakeidx}

\makeindex[intoc, title=Index]
\indexsetup{othercode={\lhead{\emph{Index}}}}

\ifx \nextra \undefined
  \usepackage[pdftex,
    hidelinks,
    pdfauthor={Chester Wong},
    pdfsubject={Imperial Maths Notes: \npart\ - \ncourse},
    pdftitle={\npart\ - \ncourse},
  pdfkeywords={Imperial Mathematics Maths Math \npart\ \nterm\ \nyear\ \ncourse}]{hyperref}
  \title{\npart\ --- \ncourse}
\else
  \usepackage[pdftex,
    hidelinks,
    pdfauthor={Chester Wong},
    pdfsubject={Imperial Maths Notes: Part \npart\ - \ncourse\ (\nextra)},
    pdftitle={\npart\ - \ncourse\ (\nextra)},
  pdfkeywords={Imperial Mathematics Maths Math \npart\ \nterm\ \nyear\ \ncourse\ \nextra}]{hyperref}

  \title{\npart\ --- \ncourse \\ {\Large \nextra}}
  \renewcommand\printindex{}
\fi

\pgfplotsset{compat=1.12}

\pagestyle{fancyplain}
\ifx \ncoursehead \undefined
\def\ncoursehead{\ncourse}
\fi

\lhead{\emph{\nouppercase{\leftmark}}}
\ifx \nextra \undefined
  \rhead{
    \ifnum\thepage=1
    \else
      \npart\ \ncoursehead
    \fi}
\else
  \rhead{
    \ifnum\thepage=1
    \else
      \npart\ \ncoursehead \ (\nextra)
    \fi}
\fi
\usetikzlibrary{arrows.meta}
\usetikzlibrary{decorations.markings}
\usetikzlibrary{decorations.pathmorphing}
\usetikzlibrary{positioning}
\usetikzlibrary{fadings}
\usetikzlibrary{intersections}
\usetikzlibrary{cd}

\newcommand*{\Cdot}{{\raisebox{-0.25ex}{\scalebox{1.5}{$\cdot$}}}}
\newcommand {\pd}[2][ ]{
  \ifx #1 { }
    \frac{\partial}{\partial #2}
  \else
    \frac{\partial^{#1}}{\partial #2^{#1}}
  \fi
}
\ifx \nhtml \undefined
\else
  \renewcommand\printindex{}
  \DisableLigatures[f]{family = *}
  \let\Contentsline\contentsline
  \renewcommand\contentsline[3]{\Contentsline{#1}{#2}{}}
  \renewcommand{\@dotsep}{10000}
  \newlength\currentparindent
  \setlength\currentparindent\parindent

  \newcommand\@minipagerestore{\setlength{\parindent}{\currentparindent}}
  \usepackage[active,tightpage,pdftex]{preview}
  \renewcommand{\PreviewBorder}{0.1cm}

  \newenvironment{stretchpage}%
  {\begin{preview}\begin{minipage}{\hsize}}%
    {\end{minipage}\end{preview}}
  \AtBeginDocument{\begin{stretchpage}}
  \AtEndDocument{\end{stretchpage}}

  \newcommand{\@@newpage}{\end{stretchpage}\begin{stretchpage}}

  \let\@real@section\section
  \renewcommand{\section}{\@@newpage\@real@section}
  \let\@real@subsection\subsection
  \renewcommand{\subsection}{\@ifstar{\@real@subsection*}{\@@newpage\@real@subsection}}
\fi
\ifx \ntrim \undefined
\else
  \usepackage{geometry}
  \geometry{
    papersize={379pt, 699pt},
    textwidth=345pt,
    textheight=596pt,
    left=17pt,
    top=54pt,
    right=17pt
  }
\fi

\ifx \nisofficial \undefined
\let\@real@maketitle\maketitle
\renewcommand{\maketitle}{\@real@maketitle\begin{center}\begin{minipage}[c]{0.9\textwidth}\centering\footnotesize These notes are not endorsed by the lecturers, and I have modified them (often significantly) after lectures. They are nowhere near accurate representations of what was actually lectured, and in particular, all errors are almost surely mine.\end{minipage}\end{center}}
\else
\fi

% Theorems
\theoremstyle{definition}
\newtheorem*{aim}{Aim}
\newtheorem*{axiom}{Axiom}
\newtheorem*{claim}{Claim}
\newtheorem*{cor}{Corollary}
\newtheorem*{conjecture}{Conjecture}
\newtheorem*{defi}{Definition}
\newtheorem*{eg}{Example}
\newtheorem*{ex}{Exercise}
\newtheorem*{fact}{Fact}
\newtheorem*{law}{Law}
\newtheorem*{lemma}{Lemma}
\newtheorem*{notation}{Notation}
\newtheorem*{prop}{Proposition}
\newtheorem*{question}{Question}
\newtheorem*{rrule}{Rule}
\newtheorem*{thm}{Theorem}
\newtheorem*{assumption}{Assumption}

\newtheorem*{remark}{Remark}
\newtheorem*{warning}{Warning}
\newtheorem*{exercise}{Exercise}

\newtheorem{nthm}{Theorem}[section]
\newtheorem{ndefi}[nthm]{Definition}
\newtheorem{nlemma}[nthm]{Lemma}
\newtheorem{nprop}[nthm]{Proposition}
\newtheorem{ncor}[nthm]{Corollary}


\renewcommand{\labelitemi}{--}
\renewcommand{\labelitemii}{$\circ$}
\renewcommand{\labelenumi}{(\roman{*})}

\let\stdsection\section
\renewcommand\section{\newpage\stdsection}

\newcommand\qedsym{\hfill\ensuremath{\square}}
% Strike through
\def\st{\bgroup \ULdepth=-.55ex \ULset}


%%%%%%%%%%%%%%%%%%%%%%%%%
%%%%% Maths Symbols %%%%%
%%%%%%%%%%%%%%%%%%%%%%%%%

% Matrix groups
\newcommand{\GL}{\mathrm{GL}}
\newcommand{\Or}{\mathrm{O}}
\newcommand{\PGL}{\mathrm{PGL}}
\newcommand{\PSL}{\mathrm{PSL}}
\newcommand{\PSO}{\mathrm{PSO}}
\newcommand{\PSU}{\mathrm{PSU}}
\newcommand{\SL}{\mathrm{SL}}
\newcommand{\SO}{\mathrm{SO}}
\newcommand{\Spin}{\mathrm{Spin}}
\newcommand{\Sp}{\mathrm{Sp}}
\newcommand{\SU}{\mathrm{SU}}
\newcommand{\U}{\mathrm{U}}
\newcommand{\Mat}{\mathrm{Mat}}

% Matrix algebras
\newcommand{\gl}{\mathfrak{gl}}
\newcommand{\ort}{\mathfrak{o}}
\newcommand{\so}{\mathfrak{so}}
\newcommand{\su}{\mathfrak{su}}
\newcommand{\uu}{\mathfrak{u}}
\renewcommand{\sl}{\mathfrak{sl}}

% Special sets
\newcommand{\C}{\mathbb{C}}
\newcommand{\CP}{\mathbb{CP}}
\newcommand{\GG}{\mathbb{G}}
\newcommand{\N}{\mathbb{N}}
\newcommand{\Q}{\mathbb{Q}}
\newcommand{\R}{\mathbb{R}}
\newcommand{\RP}{\mathbb{RP}}
\newcommand{\T}{\mathbb{T}}
\newcommand{\Z}{\mathbb{Z}}
\renewcommand{\H}{\mathbb{H}}

% Brackets
\newcommand{\abs}[1]{\left\lvert #1\right\rvert}
\newcommand{\bket}[1]{\left\lvert #1\right\rangle}
\newcommand{\brak}[1]{\left\langle #1 \right\rvert}
\newcommand{\braket}[2]{\left\langle #1\middle\vert #2 \right\rangle}
\newcommand{\bra}{\langle}
\newcommand{\ket}{\rangle}
\newcommand{\norm}[1]{\left\lVert #1\right\rVert}
\newcommand{\normalorder}[1]{\mathop{:}\nolimits\!#1\!\mathop{:}\nolimits}
\newcommand{\tv}[1]{|#1|}
\renewcommand{\vec}[1]{\boldsymbol{\mathbf{#1}}}

% not-math
\newcommand{\bolds}[1]{{\bfseries #1}}
\newcommand{\cat}[1]{\mathsf{#1}}
\newcommand{\ph}{\,\cdot\,}
\newcommand{\term}[1]{\emph{#1}\index{#1}}
\newcommand{\phantomeq}{\hphantom{{}={}}}
% Probability
\DeclareMathOperator{\Bernoulli}{Bernoulli}
\DeclareMathOperator{\betaD}{beta}
\DeclareMathOperator{\bias}{bias}
\DeclareMathOperator{\binomial}{binomial}
\DeclareMathOperator{\corr}{corr}
\DeclareMathOperator{\cov}{cov}
\DeclareMathOperator{\gammaD}{gamma}
\DeclareMathOperator{\mse}{mse}
\DeclareMathOperator{\multinomial}{multinomial}
\DeclareMathOperator{\Poisson}{Poisson}
\DeclareMathOperator{\var}{var}
\newcommand{\E}{\mathbb{E}}
\newcommand{\Prob}{\mathbb{P}}

% Algebra
\DeclareMathOperator{\adj}{adj}
\DeclareMathOperator{\Ann}{Ann}
\DeclareMathOperator{\Aut}{Aut}
\DeclareMathOperator{\Char}{char}
\DeclareMathOperator{\disc}{disc}
\DeclareMathOperator{\dom}{dom}
\DeclareMathOperator{\fix}{fix}
\DeclareMathOperator{\Hom}{Hom}
\DeclareMathOperator{\id}{id}
\DeclareMathOperator{\image}{image}
\DeclareMathOperator{\im}{im}
\DeclareMathOperator{\tr}{tr}
\DeclareMathOperator{\Tr}{Tr}
\newcommand{\Bilin}{\mathrm{Bilin}}
\newcommand{\Frob}{\mathrm{Frob}}

% Others
\newcommand\ad{\mathrm{ad}}
\newcommand\Art{\mathrm{Art}}
\newcommand{\B}{\mathcal{B}}
\newcommand{\cU}{\mathcal{U}}
\newcommand{\Der}{\mathrm{Der}}
\newcommand{\D}{\mathrm{D}}
\newcommand{\dR}{\mathrm{dR}}
\newcommand{\exterior}{\mathchoice{{\textstyle\bigwedge}}{{\bigwedge}}{{\textstyle\wedge}}{{\scriptstyle\wedge}}}
\newcommand{\F}{\mathbb{F}}
\newcommand{\G}{\mathcal{G}}
\newcommand{\Gr}{\mathrm{Gr}}
\newcommand{\haut}{\mathrm{ht}}
\newcommand{\Hol}{\mathrm{Hol}}
\newcommand{\hol}{\mathfrak{hol}}
\newcommand{\Id}{\mathrm{Id}}
\newcommand{\lie}[1]{\mathfrak{#1}}
\newcommand{\op}{\mathrm{op}}
\newcommand{\Oc}{\mathcal{O}}
\newcommand{\pr}{\mathrm{pr}}
\newcommand{\Ps}{\mathcal{P}}
\newcommand{\pt}{\mathrm{pt}}
\newcommand{\qeq}{\mathrel{``{=}"}}
\newcommand{\Rs}{\mathcal{R}}
\newcommand{\Vect}{\mathrm{Vect}}
\newcommand{\wsto}{\stackrel{\mathrm{w}^*}{\to}}
\newcommand{\wt}{\mathrm{wt}}
\newcommand{\wto}{\stackrel{\mathrm{w}}{\to}}
\renewcommand{\d}{\mathrm{d}}
\renewcommand{\P}{\mathbb{P}}
%\renewcommand{\F}{\mathcal{F}}

\newcommand{\contradiction}{%
  \ensuremath{{\Rightarrow\mspace{-2mu}\Leftarrow}}%
}


\let\Im\relax
\let\Re\relax

\DeclareMathOperator{\area}{area}
\DeclareMathOperator{\card}{card}
\DeclareMathOperator{\ccl}{ccl}
\DeclareMathOperator{\ch}{ch}
\DeclareMathOperator{\cl}{cl}
\DeclareMathOperator{\cls}{\overline{\mathrm{span}}}
\DeclareMathOperator{\coker}{coker}
\DeclareMathOperator{\conv}{conv}
\DeclareMathOperator{\cosec}{cosec}
\DeclareMathOperator{\cosech}{cosech}
\DeclareMathOperator{\covol}{covol}
\DeclareMathOperator{\diag}{diag}
\DeclareMathOperator{\diam}{diam}
\DeclareMathOperator{\Diff}{Diff}
\DeclareMathOperator{\End}{End}
\DeclareMathOperator{\energy}{energy}
\DeclareMathOperator{\erfc}{erfc}
\DeclareMathOperator{\erf}{erf}
\DeclareMathOperator*{\esssup}{ess\,sup}
\DeclareMathOperator{\ev}{ev}
\DeclareMathOperator{\Ext}{Ext}
\DeclareMathOperator{\fst}{fst}
\DeclareMathOperator{\Fit}{Fit}
\DeclareMathOperator{\Frac}{Frac}
\DeclareMathOperator{\Gal}{Gal}
\DeclareMathOperator{\gr}{gr}
\DeclareMathOperator{\hcf}{hcf}
\DeclareMathOperator{\Im}{Im}
\DeclareMathOperator{\Ind}{Ind}
\DeclareMathOperator{\Int}{Int}
\DeclareMathOperator{\Isom}{Isom}
\DeclareMathOperator{\lcm}{lcm}
\DeclareMathOperator{\length}{length}
\DeclareMathOperator{\Lie}{Lie}
\DeclareMathOperator{\like}{like}
\DeclareMathOperator{\Lk}{Lk}
\DeclareMathOperator{\Maps}{Maps}
\DeclareMathOperator{\orb}{orb}
\DeclareMathOperator{\ord}{ord}
\DeclareMathOperator{\otp}{otp}
\DeclareMathOperator{\poly}{poly}
\DeclareMathOperator{\rank}{rank}
\DeclareMathOperator{\rel}{rel}
\DeclareMathOperator{\Rad}{Rad}
\DeclareMathOperator{\Re}{Re}
\DeclareMathOperator*{\res}{res}
\DeclareMathOperator{\Res}{Res}
\DeclareMathOperator{\Ric}{Ric}
\DeclareMathOperator{\rk}{rk}
\DeclareMathOperator{\Rees}{Rees}
\DeclareMathOperator{\Root}{Root}
\DeclareMathOperator{\sech}{sech}
\DeclareMathOperator{\sgn}{sgn}
\DeclareMathOperator{\snd}{snd}
\DeclareMathOperator{\Spec}{Spec}
\DeclareMathOperator{\spn}{span}
\DeclareMathOperator{\stab}{stab}
\DeclareMathOperator{\St}{St}
\DeclareMathOperator{\supp}{supp}
\DeclareMathOperator{\Syl}{Syl}
\DeclareMathOperator{\Sym}{Sym}
\DeclareMathOperator{\vol}{vol}

\pgfarrowsdeclarecombine{twolatex'}{twolatex'}{latex'}{latex'}{latex'}{latex'}
\tikzset{->/.style = {decoration={markings,
                                  mark=at position 1 with {\arrow[scale=2]{latex'}}},
                      postaction={decorate}}}
\tikzset{<-/.style = {decoration={markings,
                                  mark=at position 0 with {\arrowreversed[scale=2]{latex'}}},
                      postaction={decorate}}}
\tikzset{<->/.style = {decoration={markings,
                                   mark=at position 0 with {\arrowreversed[scale=2]{latex'}},
                                   mark=at position 1 with {\arrow[scale=2]{latex'}}},
                       postaction={decorate}}}
\tikzset{->-/.style = {decoration={markings,
                                   mark=at position #1 with {\arrow[scale=2]{latex'}}},
                       postaction={decorate}}}
\tikzset{-<-/.style = {decoration={markings,
                                   mark=at position #1 with {\arrowreversed[scale=2]{latex'}}},
                       postaction={decorate}}}
\tikzset{->>/.style = {decoration={markings,
                                  mark=at position 1 with {\arrow[scale=2]{latex'}}},
                      postaction={decorate}}}
\tikzset{<<-/.style = {decoration={markings,
                                  mark=at position 0 with {\arrowreversed[scale=2]{twolatex'}}},
                      postaction={decorate}}}
\tikzset{<<->>/.style = {decoration={markings,
                                   mark=at position 0 with {\arrowreversed[scale=2]{twolatex'}},
                                   mark=at position 1 with {\arrow[scale=2]{twolatex'}}},
                       postaction={decorate}}}
\tikzset{->>-/.style = {decoration={markings,
                                   mark=at position #1 with {\arrow[scale=2]{twolatex'}}},
                       postaction={decorate}}}
\tikzset{-<<-/.style = {decoration={markings,
                                   mark=at position #1 with {\arrowreversed[scale=2]{twolatex'}}},
                       postaction={decorate}}}

\tikzset{circ/.style = {fill, circle, inner sep = 0, minimum size = 3}}
\tikzset{scirc/.style = {fill, circle, inner sep = 0, minimum size = 1.5}}
\tikzset{mstate/.style={circle, draw, blue, text=black, minimum width=0.7cm}}

\tikzset{eqpic/.style={baseline={([yshift=-.5ex]current bounding box.center)}}}
\tikzset{commutative diagrams/.cd,cdmap/.style={/tikz/column 1/.append style={anchor=base east},/tikz/column 2/.append style={anchor=base west},row sep=tiny}}

\definecolor{mblue}{rgb}{0.2, 0.3, 0.8}
\definecolor{morange}{rgb}{1, 0.5, 0}
\definecolor{mgreen}{rgb}{0.1, 0.4, 0.2}
\definecolor{mred}{rgb}{0.5, 0, 0}

\def\drawcirculararc(#1,#2)(#3,#4)(#5,#6){%
    \pgfmathsetmacro\cA{(#1*#1+#2*#2-#3*#3-#4*#4)/2}%
    \pgfmathsetmacro\cB{(#1*#1+#2*#2-#5*#5-#6*#6)/2}%
    \pgfmathsetmacro\cy{(\cB*(#1-#3)-\cA*(#1-#5))/%
                        ((#2-#6)*(#1-#3)-(#2-#4)*(#1-#5))}%
    \pgfmathsetmacro\cx{(\cA-\cy*(#2-#4))/(#1-#3)}%
    \pgfmathsetmacro\cr{sqrt((#1-\cx)*(#1-\cx)+(#2-\cy)*(#2-\cy))}%
    \pgfmathsetmacro\cA{atan2(#2-\cy,#1-\cx)}%
    \pgfmathsetmacro\cB{atan2(#6-\cy,#5-\cx)}%
    \pgfmathparse{\cB<\cA}%
    \ifnum\pgfmathresult=1
        \pgfmathsetmacro\cB{\cB+360}%
    \fi
    \draw (#1,#2) arc (\cA:\cB:\cr);%
}
\newcommand\getCoord[3]{\newdimen{#1}\newdimen{#2}\pgfextractx{#1}{\pgfpointanchor{#3}{center}}\pgfextracty{#2}{\pgfpointanchor{#3}{center}}}

\newcommand\qedshift{\vspace{-17pt}}
\newcommand\fakeqed{\pushQED{\qed}\qedhere}

\def\Xint#1{\mathchoice
   {\XXint\displaystyle\textstyle{#1}}%
   {\XXint\textstyle\scriptstyle{#1}}%
   {\XXint\scriptstyle\scriptscriptstyle{#1}}%
   {\XXint\scriptscriptstyle\scriptscriptstyle{#1}}%
   \!\int}
\def\XXint#1#2#3{{\setbox0=\hbox{$#1{#2#3}{\int}$}
     \vcenter{\hbox{$#2#3$}}\kern-.5\wd0}}
\def\ddashint{\Xint=}
\def\dashint{\Xint-}

\newcommand\separator{{\centering\rule{2cm}{0.2pt}\vspace{2pt}\par}}

\newenvironment{own}{\color{gray!70!black}}{}

\newcommand\makecenter[1]{\raisebox{-0.5\height}{#1}}

\mathchardef\mdash="2D

\newenvironment{significant}{\begin{center}\begin{minipage}{0.9\textwidth}\centering\em}{\end{minipage}\end{center}}
\DeclareRobustCommand{\rvdots}{%
  \vbox{
    \baselineskip4\p@\lineskiplimit\z@
    \kern-\p@
    \hbox{.}\hbox{.}\hbox{.}
  }}
\DeclareRobustCommand\tph[3]{{\texorpdfstring{#1}{#2}}}
\makeatother

\begin{document}

\maketitle
{\small
  \noindent\textbf{M1F}\\
}

\tableofcontents

\setcounter{section}{-1}

\section{Introduction}
In this course...

\section{Propositions, Sets and Numbers}
The propositions are like easy logic, and then a few sets and number concept will be discussed.

\subsection{Propositions}
\begin{defi}[Proposition]
	A \emph{proposition} is a \textbf{True} or \textbf{False} statement.
\end{defi}

\begin{eg}\leavevmode
	\begin{itemize}
		\item $2 + 2 = 4$
		\item $2 + 2 = 100000000$
		\item Fermat's Last Theorm
		\item Riemann Hypothesis
	\end{itemize}
\end{eg}

There are some propositions that we don't know they are true or false, like Riemann hypothesis. However, in \emph{classical mathematics}, mathematics of M1F, \textbf{every} proposition is either true or not. We are just not sure about some of them.

There are also some examples of things which are \textbf{not} propositions:

\begin{eg}\leavevmode
	\begin{itemize}
		\item $2 + 2$
		\item $2 = 2 = 4$
	\end{itemize}
	The first example is a number, but not proposition. It is not 'true' or 'false', it is 4.
	The second example doesn't even make sense. It is not a mathematical object.
\end{eg}

\subsection{Notation of proposition}
There are few connectives between propositions, they are \textbf{and}, \textbf{or}, \textbf{not}, \textbf{implies}, \textbf{if and only if}

\begin{defi}[And]
If $P$ and $Q$ are propositions, "$P$ \emph{and} $Q$" is a proposition and can be written as $P \land Q$. $P \land Q$ are true when \emph{both} $P$ and $Q$ are true.
\end{defi}

We can see the relation of $P \land Q$, $P$, and $Q$ by the truth table.

\begin{center}
	\begin{tabular}{|c|c|c|}
		\hline
		$P$ & $Q$ & $P \land Q$\\
		\hline
		$T$ & $T$ & $T$\\
		\hline
		$T$ & $F$ & $F$\\
		\hline
		$F$ & $T$ & $F$\\
		\hline
		$F$ & $F$ & $F$\\
		\hline
	\end{tabular}
\end{center}

\begin{eg}
$(2 + 2 = 4) \land (2 + 2 = 5)$ is false, since $2 + 2 = 5$ is false.
\end{eg}

\begin{defi}[Or]
If $P$ and $Q$ are propositions, "$P$ \emph{or} $Q$" is a proposition and can be written as $P \lor Q$. $P \lor Q$ are true when \emph{either} $P$, $Q$ or \emph{both} are true.
\end{defi}

We can see the relation of $P \lor Q$, $P$, and $Q$ by the truth table.

\begin{center}
	\begin{tabular}{|c|c|c|}
		\hline
		$P$ & $Q$ & $P \lor Q$\\
		\hline
		$T$ & $T$ & $T$\\
		\hline
		$T$ & $F$ & $T$\\
		\hline
		$F$ & $T$ & $T$\\
		\hline
		$F$ & $F$ & $F$\\
		\hline
	\end{tabular}
\end{center}

\begin{eg}
$(2 + 2 = 4) \lor (2 + 2 = 5)$ is false, since $2 + 2 = 4$ is true.
\end{eg}

\begin{defi}[Not]
If $P$ is proposition, "not $P$" is a proposition and can be written as $\neg P$. $\neg P$ is the proposition which is "the opposite of $P$". If $P$ is true then $\neg P$ is false, and if $P$ is false then $\neg P$ is true.
\end{defi}

We can see the relation of $\neg P$ and $P$ by the truth table.

\begin{center}
	\begin{tabular}{|c|c|}
		\hline
		$P$ & $\neg P$\\
		\hline
		$T$ & $F$\\
		\hline
		$F$ & $T$\\
		\hline
	\end{tabular}
\end{center}

\begin{eg}
Let $P$ be the Riemann hypothesis, then $P \lor \neg P$ is true, because in classical mathematics, the Riemann hypothesis is either true or false.
\end{eg}

\begin{defi}[Implies]
If $P$ and $Q$ are propositions, "$P$ \emph{implies} $Q$" is a proposition and can be written as $P \implies Q$. $P \implies Q$ means if $P$ is true, then $Q$ is true as well.
\end{defi}

We can see the relation of $P \implies Q$, $P$, and $Q$ by the truth table.

\begin{center}
	\begin{tabular}{|c|c|c|}
		\hline
		$P$ & $Q$ & $P \implies Q$\\
		\hline
		$T$ & $T$ & $T$\\
		\hline
		$T$ & $F$ & $F$\\
		\hline
		$F$ & $T$ & $T$\\
		\hline
		$F$ & $F$ & $T$\\
		\hline
	\end{tabular}
\end{center}

The only time that $P \implies Q$ is false is when $P$ is true and $Q$ is false.

\begin{eg}
$(2 + 2 = 4) \implies (2 + 2 = 5)$ is false, but $(2 + 2 = 5) \implies (2 + 2 = 4)$ is true.
\end{eg}

\begin{notation}
$Q \Longleftarrow P$ is defined to be $P \implies Q$.
\end{notation}

\begin{defi}[if and only if]
If $P$ and $Q$ are propositions, "$P$ \emph{if and only if} $Q$" is a proposition and can be written as $P \iff Q$. $P \iff Q$ is true when $P$ and $Q$ have the \emph{same} truth value.
\end{defi}

We can see the relation of $P \implies Q$, $P$, and $Q$ by the truth table.

\begin{center}
	\begin{tabular}{|c|c|c|}
		\hline
		$P$ & $Q$ & $P \iff Q$\\
		\hline
		$T$ & $T$ & $T$\\
		\hline
		$T$ & $F$ & $F$\\
		\hline
		$F$ & $T$ & $F$\\
		\hline
		$F$ & $F$ & $T$\\
		\hline
	\end{tabular}
\end{center}

$\iff$ is the proposition version of $=$ for numbers. If $x$ and $y$ are equal numbers, we write $x = y$, but if $P$ and $Q$ are propositions with the same truth value, we write $P \iff Q$.

\begin{eg}\leavevmode
	\begin{itemize}
		\item $(P \implies Q) \iff (Q \Longleftarrow P)$ is always true.
		\item $P \iff (\neg P)$ is always false.
	\end{itemize}
\end{eg}

\subsection{Theorm of propositions}
\begin{thm}[Relation of \emph{not, in} and \emph{or}]
Let $P$ and $Q$ be propositions,
$$(\lnot P) \lor (\lnot Q) \iff \lnot(P \land Q)$$
\end{thm}
\begin{proof}
Consider the truth table,
\begin{center}
	\begin{tabular}{|c|c|c|c|}
		\hline
		$P$ & $Q$ & $(\lnot P) \lor (\lnot Q)$ & $\lnot(P \land Q)$\\
		\hline
		$T$ & $T$ & $F \lor F \iff F$ & $\lnot(T) \iff F$\\
		\hline
		$T$ & $F$ & $F \lor T \iff T$ & $\lnot(F) \iff T$\\
		\hline
		$F$ & $T$ & $T \lor F \iff T$ & $\lnot(F) \iff T$\\
		\hline
		$F$ & $F$ & $T \lor T \iff T$ & $\lnot(F) \iff T$\\
		\hline
	\end{tabular}
\end{center}
We can see that the truth values of proposition $(\lnot P) \lor (\lnot Q)$ and $\lnot(P \land Q)$ are always the same.
$$\therefore (\lnot P) \lor (\lnot Q) \iff \lnot(P \land Q)$$
\end{proof}

\subsection{Sets}
\begin{defi}[Set]
	A \emph{set} is a collection of stuff. The things in a set $X$ are called the \emph{elements} of $X$.
\end{defi}
Note that there is a more rigorous definition of a set. The more rigorous one depends on which axiomatic foundation using for mathematics. If set theory is the foundation, the definition of a set will be "\textbf{Everything is a set.}".

\subsection{Basic notation for sets.}

\begin{notation}
We use $\{$ and $\}$ to denote sets.
\end{notation}

\begin{eg}\leavevmode
	\begin{itemize}
		\item $\{1, 2, 3\}$ is a set.
		\item $\{$ me, you, the desk in my office $\}$ is a set.
		\item $\{\}$ is a set. It exists, but it has no elements.
		\item $\{1, 2, 3, 2\}$ is a set.
		\item $\{1, 2, 3, 4, 5, ...\}$ is a set, and it is an infinite set.
	\end{itemize}
\end{eg}

We use the symbol $\in$ to denote set membership. If $a$ is a thing (e.g. a number) and $X$ is a set, then $a \in X$ is a proposition. The proposition $a \in X$ is true exactly when $a$ is in set $X$.

\begin{eg}\leavevmode
	\begin{itemize}
		\item $2 \in \{ 1, 2, 3 \}$. This means $2$ is an element of set $\{ 1, 2, 3 \}$.
		\item $x \in \{ \}$ makes mathematical sense, but it is a false statement.
	\end{itemize}
\end{eg}

\begin{notation}
$\{ \}$ has no elements, which is called the \emph{empty set}. We use $\varnothing$ to notate an empty set.
\end{notation}

\subsection{Fundamental fact about equality of sets}

\begin{defi}[Equality of sets]
\[
	X = Y \iff (\forall a \in \Omega, a \in X \iff a \in Y)
\]
It means two sets are equal if and only if they have the same elements.
\end{defi}

\begin{eg}
$\{ 1, 2, 3 \}$ and $\{ 1, 2, 3, 2 \}$ are equal.
\end{eg}

Fundamental fact above is the rule for sets. If we need to count things, we can use other things, like multisets, lists, or sequences, instead of sets.

\subsection{Notation of sets}

\subsubsection{Subsets}
\begin{notation}
We use $\subseteq$ to denote subsets. $X \subseteq Y$ is a proposition saying that $X$ is a subset of $Y$.
\end{notation}

\begin{defi}[Subset]
\[
	X \subseteq Y \iff (\forall a \in \Omega, a \in X \implies a \in Y)
\]
It means X is a subset of Y when every elements of X is also an element of Y.
\end{defi}

\begin{eg}\leavevmode
	\begin{itemize}
		\item $\{ 1, 2 \} \subseteq \{ 1, 2, 3 \}$, since elements of set $\{ 1, 2\}$, $1$ and $2$ are both inthe set $\{ 1, 2, 3 \}$.
		\item If $a$ is my left shoe, $b$ is my right hand, and $c$ is my mother, then $\{ a, b \} \subseteq \{ a, b, c \}$
	\end{itemize}
\end{eg}

\begin{notation}
$X \supseteq Y$ means $X \subseteq Y$.
\end{notation}

\begin{thm}[Equality and subsets]
If $X$ and $Y$ are sets, then
\[
	X = Y \iff (X \subseteq Y \land Y \subseteq X)
\]
\end{thm}

\begin{proof}
From $X \subseteq Y$, we can deduce 
\[
	a \in X \implies a \in Y \tag{1}
\]
And from $Y \subseteq X$, we can deduce
\[
	a \in Y \implies a \in X \tag{2}
\]
From (1) and (2), we can deduce that $$a \in Y \iff a \in X$$which is definition of $X = Y$
\[
	\therefore (X \subseteq Y \land Y \subseteq X) \implies X = Y \tag{a}
\]
Similarly, From $X = Y$, we can deduce
\[
	a \in Y \iff a \in X
\]
And it is equivalent to
\begin{align*}
	a \in Y &\implies a \in X\\
	a \in X &\implies a \in Y
\end{align*}
which are definition of $Y \subseteq X$ and $X \subseteq Y$.
\[
	\therefore X = Y \implies (X \subseteq Y \land Y \subseteq X) \tag{b}
\]
With (a) and (b), we can conclude that,
\[
	X = Y \iff (X \subseteq Y \land Y \subseteq X)
\]
\end{proof}

\subsection{Important sets}
\begin{eg}\leavevmode
	\begin{itemize}
		\item $\Z$ Integers
		\item $\Q$ Rational numbers
		\item $\R$ Real numbers
		\item $\C$ Complex numbers
	\end{itemize}
\end{eg}

\begin{defi}[Integers $\Z$]
$$\Z = \{ ..., -3, -2, -1, 0, 1, 2, 3, ...\}$$
\end{defi}
There is a problem of unconsistent of natural numbers $\N$. Someone defined it as $$\N = \{ 0, 1, 2, 3, ...\}$$Someone defined it as $$\N = \{ 1, 2, 3, ...\}$$
In M1F, we will not use $\N$. Instead, we will use the following notations.
\begin{notation}
$$\Z _{\geq 0} = \{ 0, 1, 2, 3, ...\}$$
$$\Z _{\geq 1} = \{ 1, 2, 3, ...\}$$
\end{notation}

For set $\R$, there are some special notations.
\begin{notation}
Let $a$ and $b$ be real numbers,
$$[a,b] = \{x \in \R \mid a \leq x \land x \leq b\}$$
$$(a,b) = \{x \in \R \mid a < x \land x < b\}$$
$$[a,\infty) = \{x \in \R \mid a \leq x\}$$
\end{notation}

\subsubsection{Universes}
\begin{notation}
We use \emph{universe} $\Omega$ to denote the set consisting of all the stuff we are interested in.
\end{notation}
\emph{Universe} means the set we are considering. For example, $\Omega$ could be a set of real numbers, or complex numbers. It depends on what we are considering.

\subsubsection{For all}
\begin{notation}
We use $\forall$ to say for all in mathematics.
\end{notation}

\begin{eg}
$\forall a \in \Z, 2a$ is even.\\
This means ``For all integers $a$, $2a$ is an even number''.
\end{eg}

\subsubsection{There exists}
\begin{notation}
We use $\exists$ to say there exists inn mathematics.
\end{notation}

\begin{eg}
$\exists a \in \Z, a$ is even.\\
This means ``There exists an integer $a$, which is an even number''.
\end{eg}

\subsubsection{Union}
\begin{defi}[Unions]
\[
	\forall a \in \Omega, a \in X \cup Y \iff a \in X \lor a \in Y
\]
It means the \emph{union} of $X$ and $Y$, $X \cup Y$ is all the stuff in either $X$, \emph{or} $Y$, \emph{or both}.
\end{defi}
\begin{eg}
Let $X = \{ 1, 2, 3 \}$ and $Y= \{ 3, 4, 5 \}$,\\
then $X \cup Y = \{ 1, 2, 3, 4, 5 \}$.
\end{eg}
We have some notations for intersection of large numbers of sets.\\
Let us define $I = \Z_{\geq1} = \{1,2,3,...\}$. For every $i \in I$, we have a set of real numbers $X_i \subseteq \R$.
\begin{notation}
$$\bigcup^\infty_{i=1} X_i = \{a \in \Omega \mid \exists i \in \Z_{\geq1}, a \in X_i\}$$
$$\bigcup_{i \in I} X_i = \{a \in \Omega \mid \exists i \in I, a \in X_i\}$$
\end{notation}

\begin{eg}
Let $I = \R$.
If $i \in I$, and let $X_i = \{i\}$.
What is $\bigcup_{i \in I} X_i$?
$$\bigcup_{i \in I} X_i = \{a \in \R \mid \exists i \in I, a \in X_i\}$$
\[
	\therefore \bigcup_{i \in I} X_i \subseteq \R \tag{1}
\]
Let $a \in \R$,
\[
	a \in X_a = \{a\} \tag{by definition}
\]
$\therefore \exists i \in I = \R$ such that $a \in X_i = \{i\}$ when $i = a$.
\[
	\therefore \R \subseteq \bigcup_{i \in I} X_i \tag{2}
\]
$$\bigcup_{i \in I} X_i = \R$$
\end{eg}

\subsubsection{Intersection}
\begin{defi}[Intersection]
\[
	\forall a \in \Omega, a \in X \cap Y \iff a \in X \land a \in Y
\]
It means the \emph{intersection} of $X$ and $Y$, $X \cap Y$ is all the stuff in \emph{both} $X$, \emph{and} $Y$.
\end{defi}
\begin{eg}
Let $X = \{ 1, 2, 3 \}$ and $Y= \{ 3, 4, 5 \}$,\\
then $X \cap Y = \{ 3 \}$.
\end{eg}
We have some notations for intersection of large numbers of sets.\\
Let us define $I = \Z_{\geq1} = \{1,2,3,...\}$. For every $i \in I$, we have a set of real numbers $X_i \subseteq \R$.
\begin{notation}
$$\bigcap^\infty_{i=1} X_i = \{a \in \Omega \mid \forall i \in \Z_{\geq1}, a \in X_i\}$$
$$\bigcap_{i \in I} X_i = \{a \in \Omega \mid \forall i \in I, a \in X_i\}$$
\end{notation}

\begin{eg}
What is $\bigcap_{i=1}^\infty X_i$, where $X_i = [-i, i]$?\\
$\because X_1 \subseteq X_2 \subseteq X_3 \subseteq ...$, real numbers in all the $X_i$ are the real numbers in $X_1$.\\
$\therefore \bigcap_{i=1}^\infty X_i = X_1$
\end{eg}


\subsubsection{Complements}
\begin{defi}[Complements]
\[
	\forall a \in \Omega, a \in X^c \iff \lnot(a \in X)
\]
It means if $X$ is a subset of $\Omega$, then its \emph{complement} $X^c$ is the set whose elements are all the things in $\Omega$ which are not in $X$.
\end{defi}
\begin{eg}
If our universe $\Omega$ is $\Z$, the integers, and if $X$ is the set of even integers, then its \emph{complement} $X^c$ is the set of odd numbers.
\end{eg}
\begin{notation}
$a \notin X$ is defined to be $\lnot(a \in X)$, since $a$ is not an element of $X$ is also a proposition.
\end{notation}

\subsection{Notation of sets with certain property}
Let $X$ be the set of \emph{integers}, and we want to consider the subset of $X$ consisting of positive integers. We can write the subset as: 
$$\{ a \in X \mid a > 0\}$$
The line in the middle is pronounced ``'such that". So the full statement can be read as ``the elements $a$ of $X$ such that $a > 0$".

\subsection{Theorm of sets}
\begin{thm}[A theorm of complement]
Let $X$ and $Y$ be sets.\\
If $X, Y \subseteq \Omega$, $$(X \cup Y = \Omega) \land (X \cap Y = \varnothing) \implies X = Y^c$$
\end{thm}
\begin{proof}
Let $a \in \Omega$,
$P$ be proposition $a \in X$, $Q$ be proposition $a \in Y$,
\begin{align*}
	&&a \in X \cup Y &\iff (a \in X) \lor (a \in Y) \tag{Union definition}\\
	&\therefore &a \in X \cup Y &\iff P \lor Q\\
	&\because &X \cup Y = \Omega \\
	&\therefore &a \in X \cup Y &\iff \top\\
	&&P \lor Q &\iff \top\\
	&&a \in X \cap Y &\iff (a \in X) \land (a \in Y) \tag{Intersection definition}\\
	&\therefore &a \in X \cup Y &\iff P \land Q\\
	&\because &X \cup Y = \varnothing\\
	&\therefore &a \in X \cup Y &\iff \bot\\
	&&P \land Q &\iff \bot\\
	&&\lnot (P \land Q) &\iff \top\\
	&\therefore &P \lor Q &\iff \lnot (P \land Q)\\
	&\therefore &P &\iff \lnot Q\\
	&& a \in X &\iff \lnot(a \in Y)\\
	&& a \in X &\iff a \in Y^c \tag{Complement definition}\\
	&\therefore &X &= Y^c
\end{align*}
$$(X \cup Y = \Omega) \land (X \cap Y = \varnothing) \implies X = Y^c$$
\end{proof}
Let $S = \{a \in \R \mid a > 0\}$
\begin{prop}[$S$ has a smallest element]
$$P := \exists s \in S, \forall t \in S, s \leq t$$
\end{prop}
\begin{proof}
Consider $\neg P$,
$$\neg P = \forall s \in S, \exists t \in S, s > t$$
Let $s \in S$,\\
$\frac{s}{2}$ will also be a real number, and it is smaller than $s$.\\
$\therefore \neg P$ is true, and so $P$ is a false proposition.\\
Hence, $S$ does not have a smallest statement.
\end{proof}

\subsection{Some Proof Examples}
\begin{nlemma}\label{square_int_even}
If $x$ is an integer, and $x^2$ is even, then $x$ is even.
\end{nlemma}
\begin{proof}
Assume $x$ is an integer and $x^2$ is even.\\
Assume for contradiction that x is odd.\\
Then, $x = 2t + 1$, so $x^2 = 4t^2 + 4t + 1$.\\
$x^2 = 2(2t^2 + 2t) + 1$, which is an odd number.\\
However, we assumed that $x^2$ is even at the beginning, so contradiction occurs. (\contradiction)\\
Hence, the assumption that x is odd must be wrong, so $x$ should be even.
\end{proof}

\begin{nlemma}\label{sqrt2_irr}
$\sqrt{2}$ is irrational.
\end{nlemma}
\begin{proof}
Assume for a conntradiction that $\sqrt{2}$ is rational.\\
Write $\sqrt{2} = \frac{a}{b}$, with $a,b \in \Z_{\geq1}$, and at least one of them is odd.\\
By squaring both sides, we can deduce$$2 = \frac{a^2}{b^2}$$
$$2b^2 = a^2$$
It shows that $a^2$ is an even number. By Lemma~\ref{square_int_even}, $a$ will be even.\\
Write $a = 2c$, with $c \in \Z_{\geq1}$, we can deduce$$2b^2 = (2c)^2$$
$$2b^2 = 4c^2$$
$$b^2 = 2c^2$$
Similarly, by Lemma~\ref{square_int_even}, $b$ will be even.\\
However, we assumed that one of $a,b$ is odd, so contradiction occurs.(\contradiction)\\
Hence, the assumption that $\sqrt{2}$ is rational is wrong, so $\sqrt{2}$ is irrational.
\end{proof}

\begin{nlemma}\label{Quot_Consistent}
$$a,b \in \Q \implies a+b, a-b, ab \in \Q$$
\end{nlemma}
\begin{proof}
Write $a = \frac{m}{n}$, $b = \frac{r}{s}$, with $m,n,r,s \in \Z$ annd $n,s \neq 0$\\
We can deduce, $$a \pm b = \frac{ms \pm rn}{ns}$$
Since $ms \pm rn \in \Z$ and $ns \neq 0$, therefore $a \pm b \in \Q$\\
We cann also deduce, $$ab = \frac{mr}{ns}$$
Since $mr \in \Z$ and $ns \neq 0$, therefore $ab \in \Q$\\
\end{proof}

\begin{ncor}\label{add_of_q_nq}
$$a \in \Q, b \notin \Q \implies a + b \notin \Q$$
\end{ncor}
\begin{proof}
Assume $a \in \Q, b \notin \Q$. And we also assume, $a + b \in \Q$ for contradiction.\\
We know $b = (a + b) - a$, and $a+b, a \in \Q$ by assumption.\\
By Lemma~\ref{Quot_Consistent}, $b \in \Q$.\\
However, we assumed $b \in \Q$ at the beginning. (\contradiction)\\
Hence, assumption $a + b \in \Q$ is false, so Corollary~\ref{add_of_q_nq} is proved.
\end{proof}

\begin{ncor}\label{inf_irr_num}
There are infinitely many irrational numbers.
\end{ncor}
\begin{proof}
There are infinitely many integers.
Consider $n \in \Z$, $$a = n + \sqrt{2}$$
$\sqrt{2}$ is irrational by Lemma~\ref{sqrt2_irr}, and $a \notin \Q$ by Corollary~\ref{add_of_q_nq}.\\
Thus there are infinitely many $a$.\\
Therefore, there are infinitely many irrational numbers.
\end{proof}

\section{Real Numbers}
Assume we have constructed the real numbers complete with $+, -, \times, \div$.\\
We have also proved the facts that if $a,b,c \in \R$, then
\begin{itemize}
	\item $a + b = b + a$
	\item $a(b+c) = ab + ac$
	\item $1 \times a = a$
	\item $0 \neq 1$
	\item so on
\end{itemize}
Assume we have defined $<$ on the $\R$. It means if $a,b \in \R$, we have proposition $a < b$. Lets define four axioms.
\begin{axiom}[A1]\label{A1}
$\forall a, b, t \in \R, a < b \implies a + t < b + t$
\end{axiom}
\begin{axiom}[A2]\label{A2}
$\forall a, b, c \in \R, a < b \land b < c \implies a < c$
\end{axiom}
\begin{axiom}[A3]\label{A3}
$\forall a \in \R$, exactly one of $a<0, a=0, 0<a$ is true.
\end{axiom}
\begin{axiom}[A4]\label{A4}
$\forall a, b \in \R, 0 < a \land 0 < b \implies 0 < ab$
\end{axiom}
\begin{notation}
Note that $a > b$ is defined to be $b < a$.
\end{notation}

\begin{nlemma}\label{neg_ineq}
$\forall a, b \in \R, a < b \implies -b < -a$
\end{nlemma}
\begin{proof}
Assume we have $a < b$,\\
Let $t = -a-b$,
\[
a < b \implies a + t < b + t \tag{A1}
\]
\[
a - a - b < b - a - b
\]
\[
\therefore -b < -a
\]
\end{proof}
\begin{nlemma}
$x < 0 \implies -x > 0$
\end{nlemma}
\begin{proof}
Assume we have $x < 0$\\
Let $a = x, b = 0$\\
\[
a < b \implies -b < -a \tag{Lemma~\ref{neg_ineq}}
\]
\[
\therefore 0 < -x
\]
\end{proof}
\begin{nlemma}
$a \neq 0 \implies a^2 > 0$
\end{nlemma}
\begin{proof}
Assume we have $a \neq 0$\\
By A3, one of $a^2 < 0, a^2 = 0, 0 < a^2$ is true.\\
Assume for contradiction that $a^2 = 0$.\\
$$a \neq 0 \implies a^2 \neq 0$$
Hence, contradiction occurs. (\contradiction)
Assume for contradiction that $a^2 < 0$.\\

\end{proof}

























\end{document}